\chapter{Introducción}
	\begin{tabular}{|p{6cm}|p{9cm}|}
	\hline 
	\textbf{1. Nombre de la asignatura }& Electricidad y Magnetismo \\
	\hline 
	\textbf{2. Competencias }& Plantear y solucionar problemas con base en los principios y teorías de física, química y matemáticas, a
	través del método científico para sustentar la toma de decisiones en los ámbitos científico y tecnológico.
	
 \\
       \hline 
       \textbf{3. Cuatrimestre} & Segundo \\
       \hline 
       \textbf{4. Horas Prácticas} & 13 \\
       \hline
        \textbf{5. Horas Teóricas} & 32 \\
        \hline
       \textbf{6. Horas Totales} & 45 \\
        \hline
        \textbf{7. Horas Totales por semana cuatrimestre} & 3 \\
        \hline
        \textbf{8. Objetivo de la Asignatura} & El alumno describirá el comportamiento de fenómenos eléctricos y magnéticos con base en las leyes y teorías de la física que los sustentan para comprender los principios de operación de los sistemas eléctricos. \\
        \hline
	\end{tabular}
	
\begin{tabular}{|p{9.5cm}|c|c|c|}
\hline
\multirow{2}{*}{\textbf{Unidades Temáticas}} & \multicolumn{3}{c}{\textbf{Horas}} \\


  & \textbf{Prácticas} & \textbf{Teóricas} & \textbf{Totales} \\
  \hline
 I.Principios de Electricidad y Magnetismo & 2 & 4 & 6\\
\hline
II. Electrostática & 4 & 11 & 15\\
\hline
III. Electrocinética & 4 & 11 & 15 \\
\hline
IV. Fuentes de campo magnético & 3 & 6 & 9 \\
\hline
 & 13 & 32 & 45 \\
\hline


\end{tabular}