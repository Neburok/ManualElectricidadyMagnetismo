\documentclass[oneside,12pt]{book}
\usepackage[utf8]{inputenc}
\usepackage[spanish,activeacute]{babel}
\usepackage{amsmath}
\usepackage{times}
\usepackage{graphicx}
\usepackage{afterpage}
\usepackage{fancyhdr} 
\usepackage{titlesec}
\usepackage{anysize}
\usepackage{multirow}

\usepackage{bbding}
\usepackage[dvipsnames,usenames]{color}
\usepackage{cancel}

\parindent 0em
\parskip 2ex   

\pagestyle{fancy}
\fancyhf{}
\fancyfoot[RO]{\thepage} 
\renewcommand{\chaptermark}[1]{\markboth{\textit{\thechapter. #1}}{}} 
\renewcommand{\sectionmark}[1]{\markright{\textit{\thesection. #1}}} 
\newcommand{\bigrule}{\titlerule[0.5mm]} 
\titleformat{\chapter}[display]
{\bfseries\Huge} 
{% 
 \titlerule 
 \filleft
 \Large\chaptertitlename\
 \Large\thechapter} 
{0mm} 
{\filleft} 
[\vspace{0.5mm} \bigrule] 
\titlespacing{\chapter}{0mm}{-75pt}{20pt}
\linespread{1}
\papersize{27.94 cm}{21.59 cm}
\marginsize{3cm}{2.5cm}{2.5cm}{2.5cm}
\renewcommand{\rmdefault}{phv}
\renewcommand{\sfdefault}{phv}

\begin{document}  

\renewcommand\chaptername{UNIDAD}   
\frontmatter
%%%\include{frente}
\pagestyle{empty}
%%%%%%%%%%%%%%%%%%%%%%%%%%%%%%%%

\bgroup\sffamily

\noindent
\begin{tabular}{|c|c|} 
\hline

\includegraphics[width=2.5cm,height=2.8cm]{Imagenes/LogoUTEQ}&	
\Large { UNIVERSIDAD TECNOLÓGICA DE QUERÉTARO} \\
	        &Voluntad. Conocimiento. Servicio\\
\hline	
\end{tabular}
\vspace{1.5cm}
\begin{center}
\large {Programa Educativo:}

\vspace{1cm}
TÉCNICO SUPERIOR UNIVERSITARIO EN MANTENIMIENTO ÁREA INDUSTRIAL

\vspace{2cm}
\Large{ELECTRICIDAD Y MAGNETISMO}

Manual de Asignatura
\end{center}
\vspace{2cm}

\begin{large}
Autor:

Velázquez Hernández Rubén

Fecha de publicación: Sep 2020
\end{large}
\tableofcontents
\pagestyle{fancy}
\chapter{Introducción}
	\begin{tabular}{|p{6cm}|p{9cm}|}
	\hline 
	\textbf{1. Nombre de la asignatura }& Electricidad y Magnetismo \\
	\hline 
	\textbf{2. Competencias }& Plantear y solucionar problemas con base en los principios y teorías de física, química y matemáticas, a
	través del método científico para sustentar la toma de decisiones en los ámbitos científico y tecnológico.
	
 \\
       \hline 
       \textbf{3. Cuatrimestre} & Segundo \\
       \hline 
       \textbf{4. Horas Prácticas} & 13 \\
       \hline
        \textbf{5. Horas Teóricas} & 32 \\
        \hline
       \textbf{6. Horas Totales} & 45 \\
        \hline
        \textbf{7. Horas Totales por semana cuatrimestre} & 3 \\
        \hline
        \textbf{8. Objetivo de la Asignatura} & El alumno describirá el comportamiento de fenómenos eléctricos y magnéticos con base en las leyes y teorías de la física que los sustentan para comprender los principios de operación de los sistemas eléctricos. \\
        \hline
	\end{tabular}
	
\begin{tabular}{|p{9.5cm}|c|c|c|}
\hline
\multirow{2}{*}{\textbf{Unidades Temáticas}} & \multicolumn{3}{c}{\textbf{Horas}} \\


  & \textbf{Prácticas} & \textbf{Teóricas} & \textbf{Totales} \\
  \hline
 I.Principios de Electricidad y Magnetismo & 2 & 4 & 6\\
\hline
II. Electrostática & 4 & 11 & 15\\
\hline
III. Electrocinética & 4 & 11 & 15 \\
\hline
IV. Fuentes de campo magnético & 3 & 6 & 9 \\
\hline
 & 13 & 32 & 45 \\
\hline


\end{tabular}
 
\mainmatter

\chapter{Principios de Electricidad y Magnetismo}
\paragraph{Objetivo:}
El alumno demostrará fenómenos de electricidad y magnetismo,
para determinar la potencialidad de estos en la industria.

\paragraph{Resultado de aprendizaje:}
Integrará un portafolio de evidencias con los reportes de casos prácticos que incluya: 
\begin{itemize}
	\item Los efectos que produce la electricidad:
	\begin{itemize}
		\item Transformación en calor
		\item Transformación en luz
		\item Transformación en trabajo
	\end{itemize}
	\item Los fenómenos relacionados con el magnetismo:
	\begin{itemize}
		\item Campo magnético
		\item Magnetización
		\item Método utilizado para la generación de electricidad
	\end{itemize}
	\item Conclusiones
\end{itemize}
 
\section{Electricidad y Magnetismo}
Los efectos de la electricidad son un fenómeno que desde siempre han intrigado a al a humanidad. Los griegos, por ejemplo, observaron que un material resinoso, el ámbar, atraia pequeñas particulas de paja o pequeñas motas de polvo cuando era frotado. Electron ($\varepsilon \lambda \varepsilon \kappa \rho o \nu $)
\include{electrocinetica}
\include{electrostatica}
\include{magnetismo}

\end{document}