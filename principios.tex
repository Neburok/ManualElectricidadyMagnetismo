\chapter{Principios de Electricidad y Magnetismo}
\paragraph{Objetivo:}
El alumno demostrará fenómenos de electricidad y magnetismo,
para determinar la potencialidad de estos en la industria.

\paragraph{Resultado de aprendizaje:}
Integrará un portafolio de evidencias con los reportes de casos prácticos que incluya: 
\begin{itemize}
	\item Los efectos que produce la electricidad:
	\begin{itemize}
		\item Transformación en calor
		\item Transformación en luz
		\item Transformación en trabajo
	\end{itemize}
	\item Los fenómenos relacionados con el magnetismo:
	\begin{itemize}
		\item Campo magnético
		\item Magnetización
		\item Método utilizado para la generación de electricidad
	\end{itemize}
	\item Conclusiones
\end{itemize}
 
\section{Electricidad y Magnetismo}
Los efectos de la electricidad son un fenómeno que desde siempre han intrigado a al a humanidad. Los griegos, por ejemplo, observaron que un material resinoso, el ámbar, atraia pequeñas particulas de paja o pequeñas motas de polvo cuando era frotado. Electron ($\varepsilon \lambda \varepsilon \kappa \rho o \nu $)