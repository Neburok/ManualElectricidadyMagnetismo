\chapter{Principios de Electricidad y Magnetismo}
\paragraph{Objetivo:}
El alumno demostrará fenómenos de electricidad y magnetismo,
para determinar la potencialidad de estos en la industria.

\paragraph{Resultado de aprendizaje:}
Integrará un portafolio de evidencias con los reportes de casos prácticos que incluya: 
\begin{itemize}
	\item Los efectos que produce la electricidad:
	\begin{itemize}
		\item Transformación en calor
		\item Transformación en luz
		\item Transformación en trabajo
	\end{itemize}
	\item Los fenómenos relacionados con el magnetismo:
	\begin{itemize}
		\item Campo magnético
		\item Magnetización
		\item Método utilizado para la generación de electricidad
	\end{itemize}
	\item Conclusiones
\end{itemize}
 
\section{Reseña Histórica}
Alrededor del año 600 a. C., el filósofo griego Tales de Mileto descubrió que si frotaba un trozo de la resina vegetal fósil llamada ámbar, en griego élektron, este cuerpo adquiría la propiedad de atraer pequeños objetos. Algo más tarde, otro griego, Teofrasto (310 a. C.), realizó un estudio de los diferentes materiales que eran capaces de producir fenómenos eléctricos y escribió el primer tratado sobre la electricidad.

A principios del siglo XVII comienzan los primeros estudios sobre la electricidad y el magnetismo orientados a mejorar la precisión de la navegación con brújulas magnéticas. El físico real británico William Gilbert utiliza por primera vez la palabra electricidad, creada a partir del término griego elektron (ámbar). El jesuita italiano Niccolo Cabeo analizó sus experimentos y fue el primero en comentar que había fuerzas de atracción entre ciertos cuerpos y de repulsión entre otros.

Alrededor de 1672 el físico alemán Otto von Guericke construye la primera máquina electrostática capaz de producir y almacenar energía eléctrica estática por rozamiento. Esta máquina consistía en una bola de azufre atravesada por una varilla que servía para hacer girar la bola. Las manos aplicadas sobre la bola producían una carga mayor que la conseguida hasta entonces. Francis Hawksbee perfeccionó hacia 1707 la máquina de fricción usando una esfera de vidrio.

En 1733 el francés Francois de Cisternay du Fay propuso la existencia de dos tipos de carga eléctrica, positiva y negativa, constatando que:

Los objetos frotados contra el ámbar se repelen.
También se repelen los objetos frotados contra una barra de vidrio.
Sin embargo, los objetos frotados con el ámbar atraen los objetos frotados con el vidrio.

Du Fay y Stephen Gray fueron dos de los primeros físicos eléctricos en frecuentar plazas y salones para popularizar y entretener con la electricidad. Por ejemplo, se electriza a las personas y se producen descargas eléctricas desde ellas, como en el llamado beso eléctrico: se electrificaba a una dama y luego ella daba un beso a una persona no electrificada

En 1745 se construyeron los primeros elementos de acumulación de cargas, los condensadores, llamados incorrectamente por anglicismo capacitores, desarrollados en la Universidad de Leyden (hoy Leiden) por Ewald Jürgen Von Kleist y Pieter Van Musschenbroeck. Estos instrumentos, inicialmente denominados botellas de Leyden, fueron utilizados como curiosidad científica durante gran parte del siglo XVIII. En esta época se construyeron diferentes instrumentos para acumular cargas eléctricas, en general variantes de la botella de Leyden, y otros para manifestar sus propiedades, como los electroscopios.

En 1767, Joseph Priestley publicó su obra The History and Present State of Electricity, sobre la historia de la electricidad hasta esa fecha. Este libro sería durante un siglo el referente para el estudio de la electricidad. En él, Priestley anuncia también alguno de sus propios descubrimientos, como la conductividad del carbón. Hasta entonces se pensaba que solo el agua y los metales podían conducir la electricidad.

En 1785 el físico francés Charles Coulomb publicó un tratado en el que se describían por primera vez cuantitativamente las fuerzas eléctricas, se formulaban las leyes de atracción y repulsión de cargas eléctricas estáticas y se usaba la balanza de torsión para realizar mediciones. En su honor, el conjunto de estas leyes se conoce con el nombre de ley de Coulomb. Esta ley, junto con una elaboración matemática más profunda a través del teorema de Gauss y la derivación de los conceptos de campo eléctrico y potencial eléctrico, describe la casi totalidad de los fenómenos electrostáticos.

Durante todo el siglo posterior se sucedieron avances significativos en el estudio de la electricidad, como los fenómenos eléctricos dinámicos producidos por cargas en movimiento en el interior de un material conductor. Finalmente, en 1864 el físico escocés James Clerk Maxwell unificó las leyes de la electricidad y el magnetismo en un conjunto reducido de leyes matemáticas.